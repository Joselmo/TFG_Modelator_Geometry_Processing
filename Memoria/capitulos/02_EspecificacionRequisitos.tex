\chapter{ Especificación de requisitos}

\section{Objetivos}

\begin{enumerate}[label=\textbf{\textit{OBJ-\arabic*}}]
	\item El sistema a desarrollar es un motor gráfico capaz de renderizar y mostrar elementos geométricos al usuario a partir de un fichero con formáto ply que contenga una malla de triángulos.
	\item El motor gráfico debe permitir la interacción con el usuario de una forma cómoda y agradable mediante la interfaz de usuario que muestre la malla en su estado actual y una cámara para poder visualizarla desde el ángulo deseado.
	\item Estudio e implementación de algoritmos de procesado geométrico para la simplificación de una malla mediante la reducción de aristas y caras. 
	\item Implementación de la estructura de datos de semi-aristas aladas con las operaciones necesarias para una correcta navegación e interacción con la malla de triángulos.
	\item Implementación y análisis del algoritmo de procesado geométrico ``Decimation'' con varias mallas de triángulos. Donde el usuario pueda especificar una tasa de reducción que se vaya a aplicar a la malla.
	
\end{enumerate}

\newpage
\section{ Requisitos Funcionales}


\begin{enumerate}[label=\textbf{\textit{RF-\arabic*}},ref=RF-\arabic*]
	\item \label{RF1}Almacenar mallas de triángulos.
	
	\begin{enumerate}[label*=\textbf{\textit{.\arabic*}}]
		\item \textit{Almacenar los vértices de la malla.} El modelador tiene una estructura de datos para la manipulación de los vértices \ref{RI1}, \ref{RI2}. 
		\item \textit{Almacenar las caras que componen la malla.} El modelador almacena las caras de la malla en una estructura de datos adecuada \ref{RI3}.
		\item \textit{Almacenar las aristas de la malla.} El modelador posee una estructura de datos basada en semi-aristas aladas para las aristas \ref{RI2}. 
	\end{enumerate}
	
	\item Lectura de mallas de triángulos \ref{RI3}.
		\begin{enumerate}[label*=\textbf{\textit{.\arabic*}}]
		\item \textit{Leer ficheros en formato ``Polygon File Format''.} El modelador tiene que poder leer de un fichero externo con el formato ``Polygon File Format'', sin perder la información contenida en el fichero.
		\end{enumerate}
	
	\item Interacción con la escena 3D \ref{RI3}.
	\begin{enumerate}[label*=\textbf{\textit{.\arabic*}}]
		\item \textit{Poder tener interacción con la escena 3D.} La API genera una escena donde se muestran mallas 3D y el usuario tiene que poder navegar por la escena.
	\end{enumerate}

	
	\item Mostrar información al usuario de mallas de triángulos.
	\begin{enumerate}[label*=\textbf{\textit{.\arabic*}}]
		\item \textit{Número de triangulos.} Mostrar al usuario el número de vértices que contiene la malla actual.  
		\item \textit{Número de aristas.} Mostrar al usuario el número de aristas que contiene la malla actual.
		\item \textit{Número de caras.} Mostrar al usuario el número de caras que contine la malla actual.
		
	\end{enumerate}
			
	
	\item Realizar operaciones de procesado geométrico sobre la malla.
		\begin{enumerate}[label*=\textbf{\textit{.\arabic*}}]
		\item \textit{Converger un vértice en otro.} El modelador tiene que permitir converger un vértice al vértice que apunta su semi-arista alada.
		\item \textit{Optimización de una malla.} Mediante un parámetro de reducción optimizar la malla para que alcance una reducción del número de triángulos que la componen al deseado.
		\end{enumerate}	
	
\end{enumerate}


\section{ Requisitos No Funcionales}


\begin{enumerate}[label=\textbf{\textit{RNF-\arabic*}},ref=RNF-\arabic*]
	\item Que el renderizado sea rápido.
	\item Modularizar el código para su posible reutilización.
	\item Que la interfaz sea agradable.
	\item Tiene que ser intuitivo para el usuario.
	\item La visualización de los datos ha de hacerse en tiempo real.
	\item El código ha de ser abierto.
	\item El código tiene que estar bien documentado.
	
	
	
\end{enumerate}

\section{ Requisitos de Información}


\begin{enumerate}[label=\textbf{\textit{RI-\arabic*}},ref=RI-\arabic*]
	\item \label{RI1} Vértices: Listado de vértices. Un vértice es una posición $XYZ$ y una referencia a la semi arista alada que incide en el vértice.  
	\item \label{RI2} Semi-Arista alada: 
	\item \label{RI3} Caras: Listado de 3-tuplas que hace referencia a los vércites que delimitan cada una de las caras.
	\item \label{RI4} Normales de cara: vector unitario que define a dirección de la cara \cite{llcFaceVertexNormal}.
	\item \label{RI5} Color de vértice: define el color en escala RGB de un vértice.
	%\item \label{RI4} 
	
	
	
\end{enumerate}