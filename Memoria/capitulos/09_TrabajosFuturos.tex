\chapter{ Trabajos Futuros}

En este capitulo se detallan posibles continuaciones del sistema desarrollado.

\section{ Trabajos Futuros}

Las mejoras que permite el sistema son muy amplias, al estar diseñado dentro de un área tan grande como es la informática gráfica se pueden realizar las siguientes mejoras: 

\begin{itemize}
	\item Mejorar la interfaz de usuario, permitiendo que se muestren datos de tiempo y número de elementos mediante mensajes.
	\item Se puede mejorar las opciones de visualización de la malla renderizada mediante una gestión de los buffers para que muestre la figura no solo en alambre o solido si no una combinación de ambas y un mapa de colores del error de borrar una semi-arista.
	\item Añadir otros algoritmos de procesado geométrico.
	\item Obtener la malla generada en formato .ply para guardarla en un fichero local.
\end{itemize}

En resumen, una vez tenemos un renderizador y una estructura optimizada para el procesado geométrico de mallas de triángulos, cualquier algoritmo que se desee implementar es posible, solo hay que añadirlo al sistema y generar la nueva malla o modificar la existente. 