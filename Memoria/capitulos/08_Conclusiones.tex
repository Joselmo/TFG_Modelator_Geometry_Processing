\chapter{ Conclusiones}

\section{ Conclusiones}

Partiendo de los datos arrojados en las pruebas se puede concluir que el método de decimation junto con todos los demás de los que hace uso funciona correctamente. Los datos obtenidos son los esperados por el estudio de los algoritmos y la documentación leída. Para la malla resultante del método hay dos factores que definen como va a ser el resultado.

El primero es el número de elementos, pues cuantos más elementos más opciones de simplificación se podrán realizar afectando menos a la calidad resultante.

El segundo factor es la propia definición de la malla, si la malla posee grandes superficies compuestas por muchos elementos en el mismo plano o en planos muy próximos las simplificaciones se van a centrar en esa parte. Consiguiendo así que se reduzca el número de elementos sin modificar la forma de la malla.\\

Otro factor del proyecto ha sido el estudio de los algoritmos de procesamiento geométrico de simplificación, son algoritmos muy potentes basados en ideas sencillas. Pero que para nada son triviales de implementar, requieren unas condiciones muy específicas para funcionar correctamente y ahí radica su complejidad en modificar la malla manteniendo las propiedades y condiciones deseadas.\\

En cuanto a la estructura de datos de semi-aristas aladas es un concepto que la primera vez te choca, pero una buena implementación hace que el resto de la programación de algoritmos sea mucho más fácil y precisa. Accediendo a los elementos vecinos con suma rapidez y aprovechando el espacio en memoria.\\

Por parte de la API gráfica, OpenGL ha sido muy satisfactorio trabajar con ella puesto que provee una gran cantidad de funcionalidad para que el programador sea capaz de montar estructuras en memoria según sus necesidades. La forma de manipular la geometría y vista de los Shaders ha sido muy instructiva. Y por último la implementación de un sistema completo y utilizando las tecnologías actuales ha sido muy enriquecedor.\\

