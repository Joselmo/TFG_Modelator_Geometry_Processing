\chapter*{}
%\thispagestyle{empty}
%\cleardoublepage

%\thispagestyle{empty}

\input{portada/portada_2}



\cleardoublepage
\thispagestyle{empty}

\begin{center}
{\large\bfseries Desarrollo de un motor gráfico utilizando
OpenGL/Vulkan}\\
\end{center}
\begin{center}
Jose Luis Martínez Ortiz\\
\end{center}

%\vspace{0.7cm}
\noindent{\textbf{Palabras clave}: Motor Gráfico, API gráfica, OpenGL, Procesado Geométrico, semi-aristas aladas, estructura de datos}\\

\vspace{0.7cm}
\noindent{\textbf{Resumen}}\\
El objetivo del proyecto es desarrollar una API gráfica que permita la renderización de objetos en tres dimensiones y el procesamiento geométrico de mallas de triángulos. Este procesamiento utiliza la última tecnología de las GPUs y las APIS de más bajo nivel para permitir realizar el procesamiento geométrico. Para probar la API se plantea una implementación de algoritmos de simplificación geométrica y algoritmos de deformación de mallas.
\cleardoublepage


\thispagestyle{empty}


\begin{center}
{\large\bfseries Developming Graphics Engine with OpenGL/Vulkan: Project Subtitle}\\
\end{center}
\begin{center}
Jose Luis Martínez Ortiz\\
\end{center}

%\vspace{0.7cm}
\noindent{\textbf{Keywords}: Graphic Engine, Graphic API, OpenGL, geometry processing, halfedge, data estructure}\\

\vspace{0.7cm}
\noindent{\textbf{Abstract}}\\
The objective of the present project is to develop a graphic API that allows to make geometry processing
over 3D triangular meshes, which represents objects of the real word by means of solid models concept.
The API implementation uses the last GPU technology together with a low level API in order to speed up
a great amout of operations required by the geometry processing framework. Moreover, a simplification
algorithm has been implemented using the developed API to test it.

\chapter*{}
\thispagestyle{empty}

\noindent\rule[-1ex]{\textwidth}{2pt}\\[4.5ex]

Yo, \textbf{Jose Luis Martínez Ortiz}, alumno de la titulación Grado en Ingeniería Informática de la \textbf{Escuela Técnica Superior
de Ingenierías Informática y de Telecomunicación de la Universidad de Granada}, con DNI 76636058, autorizo la
ubicación de la siguiente copia de mi Trabajo Fin de Grado en la biblioteca del centro para que pueda ser
consultada por las personas que lo deseen.

\vspace{6cm}

\noindent Fdo: Jose Luis Martínez Ortiz

\vspace{2cm}

\begin{flushright}
En Granada, a 6 de Septiembre de 2018 .
\end{flushright}


\chapter*{}
\thispagestyle{empty}

\noindent\rule[-1ex]{\textwidth}{2pt}\\[4.5ex]

D. \textbf{Alejandro José León Salas}, Profesor del Área de Lenguajes y Sistemas Informáticos del Departamento Lenguajes y Sistemas Informáticos de la Universidad de Granada.

\vspace{0.5cm}

\textbf{Informan:}

\vspace{0.5cm}

Que el presente trabajo, titulado \textit{\textbf{Desarrollo de un motor gráfico utilizando
OpenGL/Vulkan}},
ha sido realizado bajo su supervisión por \textbf{Jose Luis Martínez Ortiz}, y autorizo la defensa de dicho trabajo ante el tribunal
que corresponda.

\vspace{0.5cm}

Y para que conste, expide y firma el presente informe en Granada a 6 de Septiembre de 2018 .

\vspace{1cm}

\textbf{El director:}

\vspace{5cm}

\noindent \textbf{Alejandro José León Salas \ \ \ \ \ }

\chapter*{Agradecimientos}
\thispagestyle{empty}

       \vspace{1cm}


Me gustaría agradecer a todos los profesores y profesoras que me han ayudado durante mi etapa en la Universidad de Granada a adquirir los conocimientos que a día de hoy poseo o los que me dejaron con curiosidad para que yo continuase aprendiendo sobre la materia en cuestión. En concreto quiero agradecer su labor a Alejandro León como mi tutor y sobre todo como profesor que he tenido varios años y lo volvería tener. 

También quiero agradecer a mi familia su constante apoyo durante estos cuatro años y sobre todo esta última etapa de TFG, ya que sin ellos no hubiera sido capaz de llegar hasta donde estoy hoy.

